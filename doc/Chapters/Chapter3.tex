% Chapter 3

\chapter{Input}  % Main chapter title

\label{Chapter3}  % For referencing the chapter elsewhere, use \ref{Chapter1}

\lhead{Chapter 3. \emph{Input}}  % This is for the header on each page - perhaps a shortened title

%----------------------------------------------------------------------------------------

\section{IM3D Input Script}

IM3D input script is in the $iradina$ format, which reads the input from four input files: the first file is a general configuration file describing how the program is supposed to run (contains mostly simulation parameters etc.). The second file describes the material properties of all materials found in the target. The third file defines the structure of the simulation volume. The fourth file holds the description of the 3D target (i.e. the shapes of CSG or FETM nanosized object).

The first three files look similar to other configuration file types (and are mostly human-readable): empty lines or lines starting with a \#–sign are ignored. All other lines can either denote the beginning of a new section by [section name] or can contain a parameter definition: parname=value(s). Note that omitting any parameter might result in undefined behavior or crashes of the program. Upper and lower case letters cannot be exchanged! Recommendation: Use existing example files and adapt them according to your needs.

Alternatively, it is possible to put all the information from the four input files into one combined input file (Config.in), see details in section 3.1.5.

%----------------------------------------------------------------------------------------

\section{Input Script Structure}

\subsection{Configuration File: temp\_configfile.im3d}

The contents of the configuration file will be explained line-by-line. Detailed explanations can be found in $gen\_config$ tool (./tool/gen\_configs).

\begin{enumerate}[1:~]
\item \# Configuration file for im3d

\item ~[IonBeam]
\item max\_no\_ions=50000

\# the number of protons of the ion, range of $(1, 10^8)$, generally $10^5$ is enough and recommended;

\item ion\_Z= 14

\# the mass in atomic mass units, range of $(1, 92)$;

\item ion\_M=28.0

\# the atomic weight of ion;

\item ion\_E0=50000

\# the primary energy of the ion in unit of eV, range of $(10 eV, 2 GeV)$;

\item ion\_vx=0
\item ion\_vy=0
\item ion\_vz=1

\# the incident direction (vx, vy, vz) of the ion; range of $(0, 1)$;

\item ion\_distribution=0

\# the distribution type of the ion: 0-random, 1-centered, 2-defined position, 3-random square around predefined position, 4-Gaussian beam, sigma = bean\_spread;

\item enter\_x=20
\item enter\_y=20
\item enter\_z=-10

\# the enter position $(x, y, z)$ of ions when random beam is note selected, $z$ must be higher than the target;

\item beam\_spread=1.5

\# the spread of the beam in $xy$-space for Gaussian beam;

\item ~[Simulation]
\item OutputFileBaseName=./output/

\# output path, ./output/ is selected in default;

\item output\_format=1

\# output\_file format: 0-iradina, 1-cfg, 2-msh, 3-vtk;

\item normalize\_output=0

\# normalize output results or not;

\item display\_interval=100

\# the interval of display, range of $(1, max\_no\_ions)$;

\item storage\_interval=1000

\# the storage of display, range of $(1, max\_no\_ions)$;

\item store\_transmitted\_ions=0

\# store transmitted ions or not;

\item store\_exiting\_recoils=0

\# store existing recoils or not;

\item store\_exiting\_limit=100

\# the maximum number of exiting recoils to be stored;

\item store\_energy\_deposit=1

\# array with deposited energy are created and stored or not;

\item store\_ion\_paths=0

\# store the exact ion paths or not, =1 only for the serial version;

\item store\_path\_limit = 100;

\# the maximum number of stored paths, range of $(1, 1000)$ is recommended, =-1 when store\_ion\_paths=0;

\item transport\_type=1

\# transport type, =0 full and accurate projectile transport, =1 fast projectile transport;

\item multiple\_collisions=1

\# the maximum number of multiple collisions, 0 means just 1;

\item flight\_length\_type=0

\# the flight length type: 0->Random Poisson dist., 1->Constant;

\item flight\_length\_constant=0.3

\# if flight\_length\_type=1, set its flight\_length\_constant;

\item scattering\_calculation=0

\# the scattering calculation type: 0->SRIM-Corteo database, 1->MAGIC approximation;

\item tracing\_recoil\_or\_not=1

\# tracing the exact recoils cascades or not, 0-KP, 1-FC;

\item store\_recoil\_cascades=0

\# store the exact recoils cascades or not;

\item detailed\_sputtering=1

\# detailed calculation of sputtering or not;

\item single\_ion\_sputter\_yields=0

\# if detailed\_sputtering=1, store sputter yields for single ions or not;

\item do\_not\_store\_damage=0

\# store damages or not;

\item min\_energy=5

\# the minimum energy below which all projectiles are stopped, range of $(0, ion\_E0)$;

\item seed1=39419293
\item seed2=93145294

\# random seeds, 8-digit integers.

\item ~[Target]
\item geometry\_type=1

\# the geometry type: 0->bulk, 1->csg, 2->fetm;

\item no\_substrate=1

\# with substrate or not;

\item gen\_shape\_or\_not=1

\# if geometry\_type == 2, generate fetm shape from ply2 file by IM3D or pre-generated fetm shape by triangle.f90 code;

\item straggling\_model=3

\# the straggling model: 0->No straggling, 1->Bohr, 2->Chu, 3->Chu+Yang;

\item MaterialsFileName=Materials.in

\# filename that defines the materials in the target, in default;

\item TargetstructureFileName=Structure.in

\# filename that define the structure of the target, in default.

\end{enumerate}

\subsection{Materials File: temp\_matfile.im3d}

\begin{enumerate}[1:~]
\item ~[GaAs]

\# name;

\item element\_count=2

\# the number of elements in the material;

\item density=4.43e22

\# the density of the material;

\item elements\_Z=31,33

\# the atomic numbers of the elements in the material;

\item elements\_M=69.72,74.92

\# the atomic weights of the elements in the material;

\item elements\_conc=0.5,0.5

\# the atomic contents of the elements in the material;

\item elements\_dispEnergy=20.0,25.0

\# the displacement energies of the elements in the material;

\item elements\_latt\_energy=3.0,3.0

\# the bulk lattice energies of the elements in the material;

\item elements\_surf\_energy=2.0,1.2

\# the surface lattice energies of the elements in the material;

\item ion\_surf\_energy=2.0

\# the surface lattice energies of the ion in the material.

\end{enumerate}

\subsection{Structure File: temp\_structfile.im3d}

\begin{enumerate}[1:~]
\item \# Structure definition file for im3d

\item ~[Target]
\item cell\_count\_x=60
\item cell\_count\_y=60
\item cell\_count\_z=20

\# the numbers of cells along $(x, y, z)$-axis, respectively;

\item cell\_size\_x=10
\item cell\_size\_y=10
\item cell\_size\_z=5

\# the intervals of cells along $(x, y, z)$-axis, respectively;

\item sub\_surf\_z=101

\# the z-position of the substrate surface;

\item CompositionFileType=0

\# only used for iradina file type: 0->one column, 1->four column;

\item CompositionFileName=testwire.conc.in

\# filename that defines the composition in the target, in default.

\end{enumerate}

\subsection{Composition File: temp\_compfile.im3d}

IM3D has two different types of composition files related to $CSG$ and $FETM$ methods, respectively. 

For CSG method, explanations can be found in in $gen\_config$ tool (./tool/gen\_configs) in detail. In the composition file it includes:

~1~~~~~~~~\# Sphere\\
~~1\\
~100.000~100.000~~50.000~~50.000\\
~~0\\
~~0\\
~2~~~~~~~~\# Tetrahedron\\
~~1\\
~300.000~110.000~~~0.100\\
~300.000~~50.000~100.000\\
~250.000~140.000~100.000\\
~350.000~140.000~100.000\\
~~0\\
~~0\\
~3~~~~~~~~\# Cuboid\\
~~1\\
~450.000~~50.000~~~0.000\\
~100.000~100.000~100.000\\
~~~1.000~~~0.000~~~0.000\\
~~~0.000~~~1.000~~~0.000\\
~~~0.000~~~0.000~~~1.000\\
~4~~~~~~~~\# Ellipsoid\\
~~1\\
~~75.000~~50.000~~40.000\\
~-1\\
~100.000~300.000~~50.000\\
~~0\\
~~0\\
~~0\\
~5~~~~~~~~\# Taper\\
~~1\\
~~30.000~~80.000\\
~-1\\
~300.000~300.000~~~0.100\\
~~0\\
~~0\\
~~0\\
~6~~~~~~~~\# Column\\
~~1\\
~~50.000~100.000\\
~-1\\
~500.000~300.000~~~0.000\\
~~0\\
~~0\\
~~0\\
~7~~~~~~~~\# Polyhedron\\
~~1\\
~~6\\
~~~4\\
~~50.000~425.000~100.010\\
~~50.000~575.000~100.010\\
~~59.963~575.000~~~0.000\\
~~59.963~425.000~~~0.000\\
~~~4\\
~~59.963~425.000~~~0.000\\
~~59.963~575.000~~~0.000\\
~144.963~575.000~~~0.000\\
~144.963~425.000~~~0.000\\
~~~4\\
~144.963~425.000~~~0.000\\
~144.963~575.000~~~0.000\\
~154.926~575.000~100.010\\
~154.926~425.000~100.010\\
~~~4\\
~~50.000~425.000~100.010\\
~~50.000~575.000~100.010\\
~154.926~575.000~100.010\\
~154.926~425.000~100.010\\
~~~4\\
~~50.000~425.000~100.010\\
~~59.963~425.000~~~0.000\\
~144.963~425.000~~~0.000\\
~154.926~425.000~100.010\\
~~~4\\
~~50.000~575.000~100.010\\
~~59.963~575.000~~~0.000\\
~144.963~575.000~~~0.000\\
~154.926~575.000~100.010\\
~8~~~~~~~~\# Paraboloid\\
~~1\\
~~50.000 100.000\\
~-1\\
~300.000~500.000~~~0.000\\
~~0\\
~~0\\
~~0\\
~9~~~~~~~~\# Hyperboloid\\
~~1\\
~~20.000~~50.000~~80.000\\
~-1\\
~500.000~500.000~~~0.000\\
~~0\\
~~0\\
~~0\\
~0\\
~0.0\\

For FETM method, in the composition file it includes:

50.0~~50.0~~122.0~~~~~~~~\# box\_start\_x0, box\_start\_y0, box\_start\_z0;\\
11~~11~~11~~~~~~~~~~~~~~~\# box\_count\_x,  box\_count\_y,  box\_count\_z ;\\
10.1~~10.1~~10.1~~~~~~~~~\# box\_size\_x,   box\_size\_y,   box\_size\_z  ;\\
\# space subdivision for FETM targets;\\
2~~~~~~~~~~~~~~~~~~~~~~~~\# file format, 1-file1.dat generated by opengl, 2-file1.ply2 generated by Gmsh;\\
1~~~~~~~~~~~~~~~~~~~~~~~~\# the number of materials or file1s;\\
1~~12~~~~~~~~~~~~~~~~~~~~\# material\_type, number\_of\_data\\
100.0~~100.0~~100.0~~~~~~\# the scaling of the target, scaling\_x, scaling\_y, scaling\_z;\\
0.0~~0.0~~0.0~~~~~~~~~~~~\# the transformation of the target, trans\_x, trans\_y, trans\_z.\\

And in another input file (file1.ply2), shapes are described like (take a cube as example):

8\\
12\\
0 0 0\\
1 0 0\\
0 1 0\\
1 1 0\\
0 0 1\\
1 0 1\\
0 1 1\\
1 1 1\\
3 0 1 3\\
3 0 2 3\\
3 0 4 6\\
3 0 2 6\\
3 0 1 5\\
3 0 4 5\\
3 2 6 3\\
3 6 7 3\\
3 1 3 5\\
3 3 5 7\\
3 5 6 7\\
3 4 5 6\\

\subsection{Combined Input File}

You can also put all the information from the four input files into one combined input file, i.e., Config.in. Before running, Config.in will automatically spits into four temporary input files: temp\_configfile.im3d, temp\_matfile.im3d, temp\_structfile.im3d and temp\_compfile.im3d.
%----------------------------------------------------------------------------------------
